\documentclass{article}
\usepackage[a4paper, total={6in, 8in}]{geometry}
\usepackage[utf8]{inputenc}
\usepackage{amsmath}

\usepackage{tikz}
\usepackage{pgf}
\usepackage{pgffor}
\usepgfmodule{shapes}
\usepgfmodule{plot}
\usetikzlibrary{decorations}
\usetikzlibrary{arrows}
\usetikzlibrary{snakes}

\newcommand\setItemnumber[1]{\setcounter{enumi}{\numexpr#1-1\relax}}

% Content largely taken from previous iteration of Dordt University's PHYS232 Lab (author unknown)

\title{Measuring the Efficiency of a Stirling Engine}
\author{Instructor: Jason Ho}
\begin{document}

\maketitle
\section*{Objectives}
\begin{itemize}
    \item To observe and analyze the operation of a $\beta$-type Stirling engine.
    \item To measure the efficiency of a $\beta$-type Stirling engine by relating the energy and input of the system.
    \item To perform a linear regression analysis on a set of experimental data using binned statistics.
\end{itemize}
\section{Introduction}
A Stirling engine is a heat engine that is powered by an external source--- energy is transferred from an external reservoir of heat to fluid in the engine, which then drives pistons that can convert that heat into useful work. The single-chamber Stirling engine we will analyze in this lab is called a $\beta$-type Stirling engine (Figure \ref{}). Provided a temperature gradient can be established between the top and bottom plates of the engine (either by cooling or heating the plates), useful work can be generated.

An ideal Stirling engine follows the thermodynamic cycle represented in the $P-V$ diagram in Figure \ref{}. Starting from state 1, the Stirling cycle moves through stages of isothermal expansion at a high temperature of $T_H$, isochoric cooling, isothermal compression at a low temperature $T_C$, cycling back to its initial state through isochoric heating. 

\subsection{Theory}
In general, the efficiency $\eta$ (lowercase \textit{``eta''}) of an engine can be characterized by comparing the useful work generated by the engine compared to the net work done by the gas (\textit{i.e.,} the total work outputted by the system compared to the total energy input from the system)
\begin{equation*}
    \frac{\text{work done by engine}}{\text{work done by gas}}\equiv\eta = \frac{W_\mathrm{wheel}}{W_\mathrm{gas}}.
\end{equation*}
We can observe the energy output of the work done by the gas--- because of the work done by the gas, the wheel of the Stirling engine rotates. The energy from the heat reservoir is converted to rotational kinetic energy.
\begin{equation}
    W_\mathrm{wheel} = \frac{1}{2} I \omega^2.
\end{equation}
By approximating the wheel of the Stirling engine as a disk of radius $r$ and mass $M$, we can write
\begin{gather}
     W_\mathrm{wheel} = \frac{1}{2} \left(\frac{1}{2} M r^2 \right) (2\pi f)^2\\
     \Rightarrow  W_\mathrm{wheel} = M r^2 \pi^2 f^2
     \label{eq:work-wheel}
\end{gather}
where we have expressed the angular speed $\omega$ in terms of the frequency of rotation $f$.

For the energy input, we turn to thermodynamics. The work done by a gas $W_\mathrm{gas}$ over one rotation of the wheel is the work done over each leg of the Stirling cycle in Figure \ref{}.
\begin{equation*}
    W_\mathrm{gas} = W_{1\rightarrow 2} + W_{2\rightarrow 3} + W_{3\rightarrow 4} + W_{4\rightarrow 1},
\end{equation*}
where the work over each cycle may be determined from our expression for thermodynamic work
\begin{equation}
    W = \int P\, \mathrm{d}V.
\end{equation}
Since steps $4\rightarrow 1$ and $2\rightarrow 3$ are isochoric, there is no change in volume and 
\begin{equation*}
    W_{4\rightarrow 1} = W_{2\rightarrow 3} = 0.
\end{equation*}
For the isothermal steps, we must integrate over the changing volume while holding temperature constant. Here, we assume we can treat the gas as ideal, giving us (for isothermal expansion)
\begin{equation}
    W_{1\rightarrow 2} = \int_{V_1}^{V_2} P\, \mathrm{d}V = \int_{V_1}^{V_2} \frac{n R T_H}{V}\, \mathrm{d}V = nRT_H \ln \left( \frac{V_2}{V_1}\right) 
\end{equation}
Similarly, for isothermal compression, 
\begin{equation}
    W_{3\rightarrow 4} = \int_{V_2}^{V_1} P\, \mathrm{d}V = \int_{V_2}^{V_1} \frac{n R T_C}{V}\, \mathrm{d}V = nRT_C \ln \left( \frac{V_1}{V_2}\right),
\end{equation}
which gives us a total work done by the gas of
\begin{equation}
    W_\mathrm{gas} = W_{1\rightarrow 2} + W_{3\rightarrow 4} = n R \left(T_H - T_C \right)\ln \left(\frac{V_2}{V_1}\right)
    \label{eq:work-gas}
\end{equation}
Substituting equations \eqref{eq:work-wheel} into \eqref{eq:work-gas} and solving for $f^2$ gives us
\begin{equation}
    f^2 = \frac{nR\ln\left(V_2/V_1\right)}{M r^2 \pi^2}\eta \Delta T
    \label{eq:freq2}
\end{equation}
where $\Delta T = T_H - T_C$. By measuring the different aspects of the Stirling engine and the dependence on the frequency of the spinning wheel as the temperature difference $\Delta T$ decreases, the efficiency of the engine can be determined through a linear fit to the data.
\section{Materials}

\begin{minipage}{0.45\textwidth}
\begin{itemize}
    \item $\beta$-type Stirling Engine
    \item Vernier Photogate
    \item LabQuest
    \item $2\times$ Thermocouples
\end{itemize}
\end{minipage}
\begin{minipage}{0.45\textwidth}
\begin{itemize}
    \item Tape
    \item Kettle
    \item Water
    \item Water receptacle
\end{itemize}
\end{minipage}

\section{Procedure}
\begin{enumerate}
    % should I make the students determine their own values, or provide them with values to work with?
    \item Before we begin the engine, we will need a few values to help us calculate the efficiency later on. Measure the radius of the wheel $r$, and calculate the \textit{compression ratio} $C_R = \frac{V_2}{V_1}$ for the engine. Remember, $V_2$ represents the maximum volume of air between the plunger and the hot plate during the cycle, and $V_1$ represents the maximum volume of air between the plunger and the cold plate during the cycle.
    \item Tape the leads of the thermocouples to the top and bottom plates of the Stirling engine. These will be used to measure the temperature difference $\Delta T = T_H - T_C$. Your thermocouple may have an option to display the temperature difference $\Delta T$ on the readout.
    \item The photogate trigger on and off depending on whether the beam of light detected in the gate is interrupted by an object or now. Connect the photogate to the LabQuest. On the LabQuest, be sure the measurement setting is set to ``Pulse''. This will measure the time from when the photogate is blocked to the \textit{next} time the photogate is blocked. Determine how to convert this number into the frequency $f$ of rotation.
    \item The photogate trigger on and off depending on whether the beam of light detected in the gate is interrupted by an object or now. Connect the photogate to the LabQuest. On the LabQuest, be sure the measurement setting is set to ``Pulse''. This will measure the time from when the photogate is blocked to the \textit{next} time the photogate is blocked. Set the LabQuest to ``continuous collection'' by...
    \item Pour boiling water into a cup or a beaker and place the Stirling engine on top. Complete the next steps as the Stirling engine is heating up (within $90$ seconds).
    \item Position the photogate such that the spokes in the wheel of the Stirling engine will interrupt the signal as the wheel spins. If after about 90 seconds, the wheel does not begin to turn, give it a little push to help it overcome the static friction in the bearings. The engine should begin to run on its own.
\end{enumerate}
\subsection{Data Collection}
\begin{enumerate}
    \setItemnumber{7}
    \item The LabQuest setup will measure the frequency of the wheel automatically; your job will be to measure the  temperature difference $\Delta T = T_H - T_C$ at corresponding times at regular intervals until $\Delta T$ decreased to the point where the engine no longer runs on its own. Try to record data at every $1-2^\circ$ change in $\Delta T$. \textbf{Be sure to record the time relative to your frequency measurement so you can match up the data later on.}%If you desire, you may setup a ``Calculated Column'' in the LoggerPro interface to automatically calculate the temperature difference $\Delta T$.
    \item Plug the LabQuest into the computer and use the LoggerPro program interface to take measurements. The LabQuest will record the time and whether the photogate is blocked or unblocked--- from this you should be able to calculate the frequency $f$ of the rotation. Press ``Start'' to begin collecting the photogate data.
    \item Using the thermocouple set to register the temperature difference between the two leads ($T_2 - T_1$), measure $\Delta T$ and the time of measurement corresponding to the photogate timer at regular intervals until $\Delta T$ decreased to the point where the engine no longer runs on its own. 
    \item Repeat steps 7-9 three times. We will perform a statistical uncertainty analysis on this data.
\end{enumerate}

\subsection{Analysis}
\begin{enumerate}
    \setItemnumber{11}
    \item We will need to use \textit{data binning} here in order to perform a linear analysis on the data. Data binning is a process of dividing data into discrete groups which can be then plotted as a histogram or using some other discrete method. For the analysis, use Excel to analyze your data by hand or use the Python script mentioned by your instructor. 
    \begin{enumerate}
        \item \textbf{If using Excel}: 
        \begin{enumerate}
            \item You will likely have a large data set. Make good use of Excel formulas. Commands like \verb'MAX()', \verb'MIN()', \verb'MROUND()', \verb'SUMIF()', and \verb'COUNTIF()' may come in handy.
            \item Download the Excel template provided by your instructor. One template will be for if you want to record $T_1$ and $T_2$ separately, and the other will be for if you want to record $\Delta T$. Copy your data into the ``Data'' sheet in the workbook; be sure to arrange your columns in the following order: Time (s), Temperature(s)/$\Delta T$, and Gate State. The other sheets are locked, and you will not be able to edit them.
            \item Under the ``Analysis'' sheet, you should find your data sorted into bins of data, and corresponding average $f^2$ values. Plot $f^2$ as a function of $\Delta T$ and perform a linear regression. Use the value of the slope to calculate the efficiency of your Stirling engine using equation \eqref{eq:freq2}.
            %\item Combine your multiple experimental data sets into one table of $\Delta T$ and $f$ data. Based on your maximum and minimum values of $\Delta T$, decide how many evenly spaced temperature ``bins'' you will sort your data into. For example, if $\Delta T_\mathrm{max} = 54.4^\circ\mathrm{C}$ and $\Delta T_\mathrm{min} = 31.1^\circ\mathrm{C}$, you might try and sort your data into bins of $\Delta T = 55^\circ\mathrm{C},\,50^\circ\mathrm{C},\,45^\circ\mathrm{C},\,40^\circ\mathrm{C},\,35^\circ\mathrm{C},\,30^\circ\mathrm{C}\,$.
        \end{enumerate}
        \item \textbf{If using Python}: 
        \begin{enumerate}
            \item Using Excel, create a spreadsheet with two columns of data: the first column filled with temperature difference $\Delta T$ data, and the second column filled with frequency $f$ data. Save this as a \verb'.csv', and note what directory you saved it to.
            \item In your web browser, open up the link below to the GitHub repository that contains the Python code we will use for this analysis. Click on the ``Binder'' button. 
            \item Upload the \verb'.csv' file to the Binder directory. Use the Jupyter notebook to load the \verb'.csv' file and run a linear regression. The notebook should calculate the efficiency of the Stirling engine for you.
        \end{enumerate}
    \end{enumerate}
\end{enumerate}
\end{document}
